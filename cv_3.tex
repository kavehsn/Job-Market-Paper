%%%%%%%%%%%%%%%%%%%%%%%%%%%%%%%%%%%%%%%%%
% Freeman Curriculum Vitae
% XeLaTeX Template
% Version 2.0 (19/3/2018)
%
% This template originates from:
% http://www.LaTeXTemplates.com
%
% Authors:
% Vel (vel@LaTeXTemplates.com)
% Alessandro Plasmati
%
% License:
% CC BY-NC-SA 3.0 (http://creativecommons.org/licenses/by-nc-sa/3.0/)
%
%!TEX program = xelatex
% NOTICE: This template must be compiled with XeLaTeX, the line above should
% ensure this happens automatically but if it doesn't you will need to specify 
% XeLaTeX as the engine in your editor or script
% 
%%%%%%%%%%%%%%%%%%%%%%%%%%%%%%%%%%%%%%%%%

%----------------------------------------------------------------------------------------
%	PACKAGES AND OTHER DOCUMENT CONFIGURATIONS
%----------------------------------------------------------------------------------------

\documentclass[10pt]{article} % Font size, can be: 10pt, 11pt or 12pt

\input{structure.tex} % Include the file that specifies the document structure
% Headers and footers can be added with the \lhead{} \rhead{} \lfoot{} \rfoot{} commands
% Example right footer:
%\rfoot{\color{headings}{\sffamily Last update: \today. Typeset with Xe\LaTeX}}

%----------------------------------------------------------------------------------------

\begin{document}

\begin{paracol}{2} % Begin the multi-column environment

%----------------------------------------------------------------------------------------
%	NAME AND CURRICULUM VITAE TEXT
%----------------------------------------------------------------------------------------

\parbox[top][0.12\textheight][c]{\linewidth}{ % Parbox to hold the author name and CV text; fixed height to match the coloured box to the right, centred vertically and full line width
	\vspace{-0.04\textheight} % Reduce whitespace above the parbox to separate it from the main content
	\centering % Centre text
	{\sffamily\Huge Kaveh S. Nobari}\\\medskip % Your name
	{\Huge\color{headings}\cvtextfont Curriculum Vitae}
}

%----------------------------------------------------------------------------------------
%	MAJOR RESEARCH PROJECT
%----------------------------------------------------------------------------------------

\section{Doctoral Research}

{\raggedright\textbf{\textquotedblleft Finite-Sample Sign-Based Procedures In Linear And Non-Linear Statistical Models: With Applications To Granger Causality Analysis\textquotedblright}\\\medskip}

My thesis consists of three essays on hypothesis testing and Granger cau\-sality analysis. The main topics under consideration are: (1) exact point-optimal sign-based inference in linear and non-linear predictive regre\-ssions with a financial application; (2) sign-based measures of causa\-lity in the Granger sense with an economics application. 

\medskip % Extra whitespace before the next section

%----------------------------------------------------------------------------------------
%	WORK EXPERIENCE
%----------------------------------------------------------------------------------------

\section{Work Experience}

% Blank \workposition command to add another job:

%\workposition{} % Duration
%{} % FT/PT (full time or part time)
%{} % Employer
%{} % Job title
%{} % Description

% All 5 parameters must be supplied but any can be empty if you don't need them



%------------------------------------------------

\workposition{Oct 2016 -- July 2019} % Duration
{PT} % FT/PT (full time or part time)
{Durham University} % Employer
{Teaching Assistant } % Job title
{I was responsible for leading the seminars for the following modules:
Introduction to Financial Econometrics [2nd year UG],
Fundamentals of Finance [PG] and
Development Economics [3rd year UG]. I success\-fully main\-tained a highly
 positive feedback from both students and mo\-dule leaders throughout my three years of teaching experience, as a result of which I
 was frequently approached by the university to partake in other teaching opportunities.}  % Description

%------------------------------------------------

\workposition{Present, from Aug 2016} % Duration
{PT} % FT/PT (full time or part time)
{Aix-Marseille School of Economics} % Employer
{Researcher} % Job title
{I undertook a part-time appointment at Aix-Marseille School of Eco\-nomics, the aim of which was to assess the impact of executive cons\-traints on the vola\-tility and risk of listed firms in the MENA region. However, due to constraints imposed by my doctoral studies, the project was briefly halted until very recently. }  % Description

%------------------------------------------------

\workposition{July 2014 -- Sept 2014} % Duration
{FT} % FT/PT (full time or part time)
{Lowes Financial Management} % Employer
{Investment Analyst Intern} % Job title
{In this summer job I was tasked with assessing the performance of funds and analyzing their respective strategies. I accomplished this by emplo\-ying various methods, such as using performance ratios as well as ensu\-ring that investment objective of the funds were aligned with the needs of our clients. I was further involved with resear\-ching the Peer-to-Peer lending market in the UK, its future prospects and investment oppor\-tunities.} % Description

%------------------------------------------------

\workposition{Feb 2012 -- Aug 2012} % Duration
{FT} % FT/PT (full time or part time)
{Southern Cross Healthcare Group PLC} % Employer
{Credit Controller} % Job title
{I was involved in the solvent wind down of the Southern Cross Health\-care Group. I worked closely with the administrators of the company (namely Grant Thornton) and remained as the last contro\-ller to ensure a smooth transition of the care homes to the their new healthcare provider.} % Description

\vspace{-\baselineskip}\medskip % Standardise the whitespace after this section and before the next (the custom command adds too much otherwise)

\switchcolumn % Switch to the next paracol column

%----------------------------------------------------------------------------------------
%	COLOURED CONTACT DETAILS BOX
%----------------------------------------------------------------------------------------

\parbox[top][0.12\textheight][c]{\linewidth}{ % Parbox to hold the colour box; fixed height to match the name/CV text to the left, centred vertically and full line width
	\vspace{-0.04\textheight} % Reduce whitespace above the parbox to separate it from the main content
	\colorbox{white}{ % Create the coloured box
		\begin{supertabular}{p{0.05\linewidth}|p{0.775\linewidth}} % Start a table with two columns, the table will ensure everything is aligned
			\raisebox{-1pt}{\faHome} & 23A Graham Road, London, W4 5DR, UK \\ % Address
			\raisebox{-1pt}{\faPhone} & +44 (0) 7813 903045 \\ % Phone number
			\raisebox{0pt}{\small\faEnvelope} & \href{mailto: cnobari@gmail.com}{cnobari@gmail.com} \\ % Email address
			\raisebox{-1pt}{\small\faLinkedinSquare} & \href{https://www.linkedin.com/in/kavehnobari}{linkedin.com/in/kavehnobari} \\ % Website
			%\raisebox{-1pt}{\faGithub} & \href{https://github.com/kavehsn}{https://github.com/kavehsn} \\
			\raisebox{-1pt}{\small\faGlobe} &  \href{https://sites.google.com/view/kavehnobari}{sites.google.com/view/kavehnobari}  \\ % Website
			\raisebox{-1pt}{\small\faFlag} & British Citizen \\ % Website
			 % GitHub profile
			%\raisebox{-1pt}{\faLinkedinSquare} & \href{https://www.linkedin.com/in/username}{https://www.linkedin.com/in/username} \\ % LinkedIn profile
			% See fontawesome.pdf in the fonts folder for all icons you can use
		\end{supertabular}
	}
}

%----------------------------------------------------------------------------------------
%	EDUCATION
%----------------------------------------------------------------------------------------

\section{Education} 

% Blank \educationentry{} command to add another degree:

%\educationentry{} % Duration
%{} % Degree
%{} % Honours, achievements or distinctions (e.g. first class honours)
%{} % Department
%{} % Institution

% All 5 parameters must be supplied but any can be empty if you don't need them

%------------------------------------------------

\begin{supertabular}{ll} % Start a table with two columns, the table will ensure everything is aligned

	%------------------------------------------------
	
	\educationentry{2015 -- 2020} % Duration
	{Ph.D. in Econometrics} % Degree
	{} % Honours, achievements or distinctions (e.g. first class honours)
	{Business School} % Department
	{Durham University, UK} % Institution
	
	%------------------------------------------------
	
	\educationentry{2014 -- 2015} % Duration
	{M.Sc. in Economics \& Finance} % Degree
	{Ranked 1st in the course} % Honours, achievements or distinctions (e.g. first class honours)
	{Business School} % Department
	{Durham University, UK} % Institution

	%------------------------------------------------
	
	\educationentry{2012 -- 2013} % Duration
	{M.A. in International Financial Analysis} % Degree
	{Ranked 2nd in the course} % Honours, achievements or distinctions (e.g. first class honours)
	{Business School} % Department
	{Newcastle University, UK} % Institution
	
	%------------------------------------------------

	\educationentry{2007 -- 2011} % Duration
	{B.A. (HONS) in Accounting \& Finance} % Degree
	{} % Honours, achievements or distinctions (e.g. first class honours)
	{Business School} % Department
	{Newcastle University, UK} % Institution
	
	%------------------------------------------------

\end{supertabular}

%----------------------------------------------------------------------------------------
%	AWARDS
%----------------------------------------------------------------------------------------

\section{Awards}

% Example \tableentry{} command to add another line:

%\tableentry{Heading}{Content}{spaceafter}

% All 3 parameters must be supplied but any can be empty if you don't need them
% A "spaceafter" value in the third parameter will add some vertical space -- this is to be used between headings

%------------------------------------------------

\begin{supertabular}{ll} % Start a table with two columns, the table will ensure everything is aligned
	
	%------------------------------------------------
	
	\tableentry{2016}{\textbf{Funding Award -- Doctoral Office}}{}
	\tableentry{}{\textit{Durham University}}{spaceafter}
	
	%------------------------------------------------
	
	\tableentry{2015}{\textbf{Outstanding Performance -- Economics \& Finance}}{}
	\tableentry{}{\textit{Durham University}}{spaceafter}
	
	%------------------------------------------------
	
	\tableentry{2015}{\textbf{Best Overall Performance -- Finance Programs}}{}
	\tableentry{}{\textit{Durham University}}{spaceafter}
	
	%------------------------------------------------
\end{supertabular}




\section{Conferences} 

% Example \tableentry{} command to add another line:

%\tableentry{Heading}{Content}{spaceafter}

% All 3 parameters must be supplied but any can be empty if you don't need them
% A "spaceafter" value in the third parameter will add some vertical space -- this is to be used between headings

%------------------------------------------------

\begin{supertabular}{ll} % Start a table with two columns, the table will ensure everything is aligned
	
	%------------------------------------------------
	
	\tableentry{2019}{Oral Presentation in Rabat (Morocco)}{}
	\tableentry{}{The Econometric Society - Africa}{spaceafter}
	
	%------------------------------------------------
		
	\tableentry{2018}{Oral Presentation in Cotonou (Benin)}{}
	\tableentry{}{The Econometric Society - Africa}{spaceafter}

	%------------------------------------------------
	
	\tableentry{2016}{Oral Presentation in Seville (Spain)}{}
	\tableentry{}{Computational and Financial Econometrics}{spaceafter}

	
\end{supertabular}


%----------------------------------------------------------------------------------------
%	COMMUNICATION SKILLS
%----------------------------------------------------------------------------------------
%----------------------------------------------------------------------------------------
%	PUBLICATIONS
%----------------------------------------------------------------------------------------

\section{Working Papers}

% Example \longformdescription{} command to add another publication:

%\longformpublication{Reference (format this manually as desired)}

%------------------------------------------------

\longformpublication{\textbf{Dufour, J.-M., Taamouti, A., Nobari, K.} Exact point-optimal sign-based tests for predictive linear and non-linear regressions \textit{Working Paper},}

\longformdescription{\textit{\textquotedblleft Predictors of stock returns are often highly persistent with innovations that are correlated
 with the disturbances in the predictive regression of returns, which leads to invalid infe\-rence using the conventional tests.
We propose point-optimal sign-based tests in the context of linear and nonlinear models that are valid
 in the presence of stochastic regressors. The proposed tests are exact,
distribution-free, and robust against hetero\-skedasticity of unknown form. Further, they
may be inverted to build confidence regions for the parameters of the
regression function.\textquotedblright}}

\switchcolumn

\longformpublication{\textbf{Nobari, K.} Pair-copula constructions of point-op\-timal sign-based tests for predictive linear and non-linear regressions \textit{Working Paper},}

\longformdescription{\textit{\textquotedblleft We extend the flexibility of the exact point-optimal sign-based tests by considering the entire dependence structure of the signs and building feasible test statistics based on pair copula constructions of the sign process. In a Monte Carlo study, we compare the performance of the proposed tests based on pair copula constructions by comparing its size and power to those of certain existing tests that are intended to be robust against heteroskedasticity. The simulation results maintain the superiority of our procedures to existing popular tests.\textquotedblright}}

\longformpublication{\textbf{Nobari, K.} Sign-based measures and tests of Granger causality \textit{Working Paper},}

\longformdescription{\textit{\textquotedblleft We propose sign-based measures of Granger causality based on the Kullback-Leibler distance that quantify the degree of causalities. Furthermore, we show that by using bound-type procedures, Granger non-causality tests between random va\-ri\-ables can be developed as a byproduct of the sign-based measures. The tests are exact, distribution-free and robust against hetero\-skedasticity of unknown form. We further suggest the VAR sieve bootstrap to reduce the bias and obtain bias-corrected estimators.  A Monte Carlo simulation study reveals that the bootstrap bias-corrected estimator of the causality measures produce the desired outcome. Fur\-ther\-more, the tests of Granger non-causality based on the signs perform well in terms of size control and power.\textquotedblright}}


% As an alternative to a long-form publication list, you can create a shorter summary using only DOI values and years.

% Example \doipublication{} command to add another publication:

%\doipublication{Year}{DOI}{firstauthor}{spaceafter}

% All four parameters are required (can be empty though)
% A value of "firstauthor" in the third parameter will print the DOI in bold
% A "spaceafter" value in the fourth parameter will add some vertical space -- this is to be used between years

%------------------------------------------------
%----------------------------------------------------------------------------------------
%	SKILLS DESCRIPTION
%----------------------------------------------------------------------------------------

\section{Additional Skills}

% Example \longformdescription{} command to add another section:

%\longformdescription{Heading}{Description}

%------------------------------------------------

\begin{supertabular}{ll} % Start a table with two columns, the table will ensure everything is aligned

	%------------------------------------------------
	
	\tableentry{Certificates}{Passed CFA Level II Exam -- CFA Institute}{} 
	\tableentry{}{Qualified Financial Risk Manager -- GARP}{spaceafter} 

	
	%------------------------------------------------
	
	\tableentry{Languages}{Fluent in English, Persian and Russian}{} 
	\tableentry{}{with an intermediate knowledge of Italian}{spaceafter} 
	
	\tableentry{Interests}{Played the piano for 7 years and received}{} 
	\tableentry{}{brief vocal training at the Rimsky-Korsakov}{}
	\tableentry{}{conservatoire in St. Petersburg (Russia).}{} 
	\tableentry{}{In addition, I enjoy cooking and attending }{}
	\tableentry{}{classical concerts in my spare time.}{spaceafter}
\end{supertabular}

%----------------------------------------------------------------------------------------
%	COMPUTER SKILLS
%----------------------------------------------------------------------------------------

\section{Computer Literacy} 

% Example \tableentry{} command to add another line:

%\tableentry{Heading}{Content}{spaceafter}

% All 3 parameters must be supplied but any can be empty if you don't need them
% A "spaceafter" value in the third parameter will add some vertical space -- this is to be used between headings

%------------------------------------------------

\begin{supertabular}{ll} % Start a table with two columns, the table will ensure everything is aligned
	
	%------------------------------------------------
	
	\tableentry{Working Knowledge}{Python, Microsoft Office,}{}
	\tableentry{}{Computer Hardware \& Support}{spaceafter}
	
	%------------------------------------------------
	
	\tableentry{Working Experience}{Matlab, R, \LaTeX, Linux, VBA,}{}
	\tableentry{}{High Performance Computing}{}
	\tableentry{}{(SLURM), Stata, Eviews, SPSS}{spaceafter}

	
\end{supertabular}


\switchcolumn
%----------------------------------------------------------------------------------------
%	REFERENCES
%----------------------------------------------------------------------------------------

\section{References}

%\textit{References available on request}

%------------------------------------------------

% Example \tableentry{} command to add another line:

%\tableentry{Heading}{Content}{spaceafter}

% All 3 parameters must be supplied but any can be empty if you don't need them
% A "spaceafter" value in the third parameter will add some vertical space -- this is to be used between headings

%------------------------------------------------

\begin{supertabular}{ll} % Start a table with two columns, the table will ensure everything is aligned
	
	%------------------------------------------------
	
	\tableentry{}{\textbf{Professor. Abderrahim Taamouti}}{spaceafter}
	\tableentry{Position}{Professor in Economics}{}
	\tableentry{Employer}{\href{https://www.dur.ac.uk/business/}{Business School}}{}
	\tableentry{}{\href{https://www.dur.ac.uk/}{\textit{Durham University}}}{spaceafter}
	\tableentry{Email}{\href{mailto:abderrahim.taamouti@durham.ac.uk}{abderrahim.taamouti@durham.ac.uk}}{}
	\tableentry{Phone}{+44 (0) 191 33 45423 (Work)}{}
	\tableentry{Webpage}{\href{https://www.dur.ac.uk/research/directory/staff/?mode=staff\&id=12888}{Prof. Abderrahim Taamouti}}{spaceafter}

	%------------------------------------------------
	
	\tableentry{}{}{} % Creates some additional whitespace between the references
	
	%------------------------------------------------

	
	\tableentry{}{\textbf{Dr. Majid Al Sadoon}}{spaceafter}
	\tableentry{Position}{Associate Professor in Economics}{}
	\tableentry{Employer}{\href{https://www.dur.ac.uk/business/}{Business School}}{}
	\tableentry{}{\href{https://www.dur.ac.uk/}{\textit{Durham University}}}{spaceafter}
	\tableentry{Email}{\href{mailto:majid.al-sadoon@durham.ac.uk}{majid.al-sadoon@durham.ac.uk}}{}
	\tableentry{Phone}{+44 (0) 191 33 47164 (Work)}{}
	\tableentry{Webpage}{\href{https://www.dur.ac.uk/research/directory/staff/?mode=staff\&id=17332}{Dr. Majid Al Sadoon}}{spaceafter}

	%------------------------------------------------
	
	\tableentry{}{}{} % Creates some additional whitespace between the references
	
	%------------------------------------------------

	\tableentry{}{\textbf{Professor. Jose Olmo}}{spaceafter}
	\tableentry{Position}{Professor in Financial Economics}{}
	\tableentry{Employer}{\href{https://www.southampton.ac.uk/economics/}{Economics Department}}{}
	\tableentry{}{\href{https://www.southampton.ac.uk/}{\textit{Southampton University}}}{spaceafter}
	\tableentry{Email}{\href{mailto: joseolmo@unizar.es}{ joseolmo@unizar.es}}{}
	\tableentry{Phone}{ +34 876 55 4682 (Work)}{}
	\tableentry{Webpage}{\href{https://sites.google.com/site/joseolmobadenas/}{Professor. Jose Olmo}}{}
	
	%------------------------------------------------
	

	
\end{supertabular}




% Example \tableentry{} command to add another line:

%\tableentry{Heading}{Content}{spaceafter}

% All 3 parameters must be supplied but any can be empty if you don't need them
% A "spaceafter" value in the third parameter will add some vertical space -- this is to be used between headings

%------------------------------------------------
%------------------------------------------------



%----------------------------------------------------------------------------------------

\end{paracol}

%----------------------------------------------------------------------------------------

\end{document}
